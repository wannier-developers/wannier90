%!TEX root=./user_guide.tex
\chapter*{Introduction}
\section*{Getting Help}
The latest release of \wannier\ and documentation can always
be found at \url{http://www.wannier.org}.

The development version may be cloned/downloaded from the
official repository of the \wannier\ code on GitHub 
(see \url{https://github.com/wannier-developers/wannier90}).

There is a \wannier\ mailing list for discussing issues in the
development, theory, coding and algorithms pertinent to MLWF.
You can register for this mailing list by following the links at
\url{http://www.wannier.org/forum.html}.
Alternatively, for technical issues about the \wannier\ code, 
check the official repository of \wannier\ on GitHub 
where you may raise issues or ask questions about about its
functionalities.

Finally, many frequently asked questions are answered in
Appendix~\ref{chap:faq}. An expanded FAQ session may be
found on the Wiki page of the GitHub repository at 
\url{https://github.com/wannier-developers/wannier90/wiki/FAQ}. 

\section*{Citation}
We ask that you acknowledge the use of \wannier\ in any publications
arising from the use of this code through the following reference
\begin{quote}
  [ref] G. Pizzi, V. Vitale, R. Arita, S. Bl\"ugel, F. Freimuth, G. G\'eranton, 
  M. Gibertini, D. Gresch, C. Johnson, T. Koretsune, J. Iba\~nez-Azpiroz, 
  H. Lee, J.M. Lihm, D. Marchand, A. Marrazzo, Y. Mokrousov, J.I. Mustafa, 
  Y. Nohara, Y. Nomura, L. Paulatto, S. Ponc\'e, T. Ponweiser, J. Qiao, 
  F. Th\"ole, S.S. Tsirkin, M. Wierzbowska, N. Marzari, D. Vanderbilt, 
  I. Souza, A.A. Mostofi, J.R. Yates,\\
  Wannier90 as a community code: new features and 
    applications, \emph{J. Phys. Cond. Matt.} {\bf 32}, 165902 (2020)\\
    \url{https://doi.org/10.1088/1361-648X/ab51ff}
  \end{quote}
  
If you are using versions 2.x of the code, cite instead:  
\begin{quote}
[ref] A.~A.~Mostofi, J.~R.~Yates, G.~Pizzi, Y.-S.~Lee, I.~Souza, D.~Vanderbilt
and N.~Marzari,\\
An updated version of \wannier: 
A Tool for Obtaining Maximally-Localised Wannier
  Functions, {\it Comput. Phys. Commun.} {\bf 185}, 2309 (2014)\\
\url{http://dx.doi.org/10.1016/j.cpc.2014.05.003}

%%\url{http://www.wannier.org/}
\end{quote}                                                              

It would also be appropriate to cite the original articles:\\\\
Maximally localized generalized Wannier functions for composite
  energy bands,\\ 
N. Marzari and D. Vanderbilt, {\it Phys. Rev. B} {\bf 56}, 12847 (1997)\\\\
Maximally localized Wannier functions for entangled energy bands,\\
I. Souza, N. Marzari and D. Vanderbilt, {\it Phys. Rev. B} {\bf 65}, 035109 (2001)


\section*{Credits}
The Wannier90 Developer Group includes Giovanni Pizzi (EPFL, CH), 
Valerio Vitale (Cambridge, GB),
David Vanderbilt  (Rutgers University, US),
Nicola Marzari    (EPFL, CH),
Ivo Souza         (Universidad del Pais Vasco, ES),
Arash A. Mostofi  (Imperial College London, GB), and 
Jonathan R. Yates (University of Oxford, GB).

The present release of \wannier\ was written by the Wannier90 Developer Group together
with Ryotaro Arita (Riken and U. Tokyo, JP),
Stefan Bl\"ugel (FZ  J\"ulich, DE),
Frank Freimuth (FZ  J\"ulich, DE),
Guillame G\'eranton (FZ  J\"ulich, DE),
Marco Gibertini (EPFL and University of Geneva, CH),
Dominik Gresch (ETHZ, CH),
Charles Johnson (Imperial College London, GB),
Takashi Koretsune (Tohoku University and JST PRESTO, JP),
Julen Iba\~nez-Azpiroz (Universidad del Pais Vasco, ES),
Hyungjun Lee (EPFL, CH),
Jae-Mo Lihm (Seoul National University, KR),
Daniel Marchand (EPFL, CH),
Antimo Marrazzo (EPFL, CH),
Yuriy Mokrousov (FZ  J\"ulich, DE),
Jamal I. Mustafa (UC Berkeley, USA),
Yoshiro Nohara (Tokyo, JP),
Yusuke Nomura (U. Tokyo, JP),
Lorenzo Paulatto (Sorbonne Paris, FR),
Samuel Ponc\'e (Oxford University, GB),
Thomas Ponweiser (RISC Software GmbH, AT),
Florian Th\"ole (ETHZ, CH),
Stepan Tsirkin (Universidad del Pais Vasco, ES),
Ma\l{}gorzata Wierzbowska (Polish Academy of Science, PL).

Contributors to the code include:
Daniel Aberg (w90pov code), 
Lampros Andrinopoulos (w90vdw code),
Pablo Aguado Puente (gyrotropic routines),
Raffaello Bianco (k-slice plotting),
Marco Buongiorno Nardelli (dosqc v1.0 subroutines upon which transport.f90 is based),
Stefano De Gironcoli (pw2wannier90.x interface to Quantum ESPRESSO),
Pablo Garcia Fernandez (matrix elements of the position operator),
Nicholas D. M. Hine (w90vdw code),
Young-Su Lee (specialised Gamma point routines and transport),
Antoine Levitt (preconditioning),
Graham Lopez (extension of pw2wannier90 to add terms needed for orbital magnetisation),
Radu Miron (constrained centres),
Nicolas Poilvert (transport routines),
Michel Posternak (original plotting routines),
Rei Sakuma (Symmetry-adapted Wannier functions),
Gabriele Sclauzero (disentanglement in spheres in k-space),
Matthew Shelley (transport routines),
Christian Stieger (routine to print the U matrices),
David Strubbe (various bugfixes/improvements),
Timo Thonhauser (extension of pw2wannier90 to add terms needed for orbital magnetisation), Junfeng Qiao (spin Hall conductivity).

We also acknowledge individuals not already mentioned above who participated in the first Wannier90 community meeting (San Sebastian, 2016) for useful discussions:
Daniel Fritsch,
Victor Garcia Suarez,
Jan-Philipp Hanke,
Ji Hoon Ryoo,
J\"urg Hutter,
Javier Junquera,
Liang Liang,
Michael Obermeyer,
Gianluca Prandini,
Paolo Umari.

\wannier\ Version 2.x was written by:
Arash A. Mostofi, Giovanni Pizzi, Ivo Souza, Jonathan R. Yates.
\wannier\ Version 1.0 was written by:
Arash A. Mostofi, Jonathan R. Yates, Young-Su Lee.
\wannier\ is based on the Wannier Fortran 77 code written for isolated bands by Nicola Marzari
and David Vanderbilt and for entangled bands by Ivo Souza, Nicola Marzari,
and David Vanderbilt.

\wannier\ \copyright\ 2007-2020 The Wannier Developer Group and individual contributors

\section*{Licence}
All the material in this distribution is free software; you can
redistribute it and/or 
modify it under the terms of the GNU General Public License
as published by the Free Software Foundation; either version 2
of the License, or (at your option) any later version.

This program is distributed in the hope that it will be useful,
but WITHOUT ANY WARRANTY; without even the implied warranty of
MERCHANTABILITY or FITNESS FOR A PARTICULAR PURPOSE.  See the
GNU General Public License for more details.

You should have received a copy of the GNU General Public License
along with this program; if not, write to the Free Software
Foundation, Inc., 51 Franklin Street, Fifth Floor, Boston, MA  02110-1301, USA.


 
