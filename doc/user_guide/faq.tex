%!TEX root=./user_guide.tex
\chapter{Frequently Asked Questions}\label{chap:faq}


\section{General Questions}

\subsection{What is \wannier?}

\wannier\ is a computer package, written in Fortran90, for obtaining
maximally-localised Wannier functions, using them to calculate
bandstructures, Fermi surfaces, dielectric properties, sparse
Hamiltonians and many things besides.

\subsection{Where can I get \wannier?}

The most recent release of \wannier\ is always available on our
website \url{http://www.wannier.org}.

\subsection{Where can I get the most recent information about
  \wannier?}

The latest news about \wannier\ can be followed on our website \url{http://www.wannier.org}.

\subsection{Is \wannier\ free?}

Yes! \wannier\ is available for use free-of-charge under the GNU
General Public Licence. See the file {\tt LICENSE} that comes with the
\wannier\ distribution or the GNU hopepage at \url{http://www.gnu.org}. 

\section{Getting Help}

\subsection{Is there a Tutorial available for \wannier?}

Yes! The {\tt examples} directory of the \wannier\ distribution
contains input files for a number of tutorial calculations. The {\tt
  doc} directory contains the accompanying tutorial handout. 

\subsection{Where do I get support for \wannier?}

There are a number of options:

\begin{enumerate}
\item The \wannier\ User Guide, available in the {\tt doc} directory of the
  distribution, and from the webpage (\url{http://www.wannier.org/user\_guide.html})
\item The \wannier\ webpage for the most recent announcements (\url{http://www.wannier.org})
\item The \wannier\ mailing list (see \url{http://www.wannier.org/forum.html})
\end{enumerate}

\subsection{Is there a mailing list for \wannier?}

Yes! You need to register: go to \url{http://www.wannier.org/forum.html} and
follow the instructions. 

\section{Providing Help: Finding and Reporting Bugs}

\subsection{I think I found a bug. How do I report it?}

\begin{itemize}
\item Check and double-check. Make sure it's a bug.
\item Check that it is a bug in \wannier\ and not a bug in the
  software interfaced to \wannier.
\item Check that you're using the latest version of \wannier.
\item Send us an email. Make sure to
  describe the problem and to attach all input and output files
  relating to the problem that you have found.
\end{itemize}

\subsection{I have got an idea! How do I report a wish?}

We're always happy to listen to suggestions. Email your idea to the
  \wannier\ developers.
% at {\tt developers@wannier.org}.

\subsection{I want to help! How can I contribute to \wannier?}

Great! There's always plenty of functionality to add. Email us 
% at {\tt developers@wannier.org} 
to let us know about the functionality you'd like to contribute. 

\subsection{I like \wannier! Should I donate anything to its authors?}

Our Swiss bank account number is... just kidding! There is no need to
donate anything, please just cite our paper in any publications that
arise from your use of \wannier:

\begin{quote}
[ref] G. Pizzi, V. Vitale, R. Arita, S. Bl\"ugel, F. Freimuth, G. G\'eranton, 
   M. Gibertini, D. Gresch, C. Johnson, T. Koretsune, J. Iba\~nez-Azpiroz, 
   H. Lee, J.M. Lihm, D. Marchand, A. Marrazzo, Y. Mokrousov, J.I. Mustafa, 
   Y. Nohara, Y. Nomura, L. Paulatto, S. Ponc\'e, T. Ponweiser, J. Qiao, 
   F. Th\"ole, S.S. Tsirkin, M. Wierzbowska, N. Marzari, D. Vanderbilt, 
   I. Souza, A.A. Mostofi, J.R. Yates,\\
   Wannier90 as a community code: new features and 
  applications, \emph{arXiv:1907.09788} (2019)\\
  \url{https://arxiv.org/abs/1907.09788}
\end{quote}

If you are using versions 2.x of the code, cite instead:

\begin{quote}
[ref] A.~A.~Mostofi, J.~R.~Yates, G.~Pizzi, Y.-S.~Lee, I.~Souza, D.~Vanderbilt
and N.~Marzari,\\
An updated version of \wannier: 
A Tool for Obtaining Maximally-Localised Wannier
  Functions, {\it Comput. Phys. Commun.} {\bf 185}, 2309 (2014)\\
\url{http://dx.doi.org/10.1016/j.cpc.2014.05.003}
\end{quote} 

\section{Installation}

\subsection{How do I install \wannier?\label{sec:installation}}

Follow the instructions in the file {\tt README.install} in the main
directory of the \wannier\ distribution.

\subsection{Are there \wannier\ binaries available?}

Not at present.

\subsection{Is there anything else I need?}

Yes. \wannier\ works on top of an electronic structure
calculation. 

At the time of writing there are public, fully functioning, interfaces
between \wannier\ and \pwscf, {\sc abinit}
(\url{http://www.abinit.org}), {\sc siesta}
(\url{http://www.icmab.es/siesta/}), {\sc VASP}
(\url{https://www.vasp.at}), {\sc Wien2k} (\url{http://www.wien2k.at}),
{\sc fleur} (\url{http://www.fleur.de}), {\sc OpenMX} (\url{http://www.openmx-square.org/}), 
{\sc GPAW} (\url{https://wiki.fysik.dtu.dk/gpaw/}).

To use 
\wannier\ in combination with 
\pwscf\ code (a plane-wave, pseudopotential, density-functional theory
code, which is part of the {\tt quantum-espresso} package) you 
will need to download \pwscf\ from the webpage \url{http://www.quantum-espresso.org}. Then compile \pwscf\
and the \wannier\ interface program {\tt pw2wannier90}. For
instructions, please refer to the
documentation that comes with the {\tt quantum-espresso} distribution.

For examples of how to use \pwscf\ and \wannier\ in conjunction with
each other, see the \wannier\ Tutorial.




%\section{Compile-time Problems}

%\section{Run-time Problems}

%\section{Using \wannier}
