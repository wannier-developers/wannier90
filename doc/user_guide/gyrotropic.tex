%!TEX root=./user_guide.tex
\chapter{Overview of the {\tt gyrotropic} module \label{ch:gyrotropic}}


The {\tt gyrotropic} module of {\tt postw90} is called by setting {\tt
  gyrotropic = true} and choosing one or more of the available options for
{\tt gyrotropic\_task}. The module computes the quantities, studied in 
 \cite{tsirkin-arxiv17}, where more details may be found.

\section{{\tt berry\_task=-d0}: the Berry curvature dipole  }

The traceless dimensionless tensor
\begin{equation}
\label{eq:D_ab}
D_{ab}=\int\dk\sum_n
\frac{\partial E_n}{\partial{k_a}}
\Omega_n^b
\left(-\frac{\partial f_0}{\partial E}\right)_{E=E_n},
\end{equation}


\section{{\tt berry\_task=-dw}: the finite-frequency generalization of the Berry curvature dipole  }

\begin{equation}
\label{eq:D-tilde}
\widetilde{D}_{ab}(\ww)=\int\dk\sum_n
\frac{\partial E_n}{\partial{k_a}}\widetilde\Omega^b_n(\ww)
\left(-\frac{\partial f_0}{\partial E}\right)_{E=E_n},
\end{equation}

where $\widetilde{\bm\Omega}_{\kk n}(\ww)$ is a finite-frequency
generalization of the Berry curvature:
%
%
\begin{equation}
\label{eq:curv-w}
\widetilde{\bm\Omega}_{\kk n}(\ww)=-
\sum_m\,\frac{\ww^2_{\kk mn}}{\ww^2_{\kk mn}-\ww^2}
\im\left({\bm A}_{\kk nm}\times{\bm A}_{\kk mn}\right)
\end{equation}
Contrary to the Berry
  curvature, the divergence of $\tilde{\bm\Omega}_{\kk n}(\ww)$ is
  generally nonzero. As a result, $\wt D(\ww)$ 
can have a nonzero trace at finite frequencies, $\tilde{D}_\|\neq-2\tilde{D}_\perp$ in Te.

\section{{\tt berry\_task=-C}: the ohmic  conductivity }

In the constant relaxation-time
approximation  the ohmic conductivity is expressed as
 $\sigma_{ab}=(2\pi e\tau/\hbar)C_{ab}$,  with
%
\beq
\label{eq:C_ab}
C_{ab}=\frac{e}{h}\int\dk\sum_n\,
\frac{\partial E_n}{\partial{k_a}} \frac{\partial E_n}{\partial{k_b}}
\left(-\frac{\partial f_0}{\partial E}\right)_{E=E_n}
\eeq
a positive quantity with
units of surface current density (A/cm).


\section{{\tt berry\_task=-K}: the kinetic magnetoelectric effect (kME) }

A microscopic theory of the intrinsic kME effect in bulk crystals was
recently developed~\cite{yoda-sr15,zhong-prl16}.  

The response is described by
\beq
\label{eq:K_ab}
K_{ab}=\int\dk\sum_n\frac{\partial E_n}{\partial{k_a}} m_n^b 
\left(-\frac{\partial f_0}{\partial E}\right)_{E=E_n},
\eeq
%
which has the same form as \eq{D_ab}, but with the Berry
curvature replaced by the intrinsic magnetic moment ${\bm m}_{\kk n}$
of the Bloch electrons, which has the  spin and orbital components
 given by~\cite{xiao-rmp10} 
%
\bea
\label{eq:m-spin}
m^{\rm spin}_{\kk n}&=&-\frac{1}{2}g_s\mu_{\rm B} \me{\psi_{\kk
      n}}{\bf \sigma}{\psi_{\kk n}}\\
\label{eq:m-orb}
{\bm m}^{\rm orb}_{\kk n}&=&\frac{e}{2\hbar}\im
\bra{{\bm\partial}_\kk u_{\kk n}}\times
(H_\kk-E_{\kk n})\ket{{\bm\partial}_\kk u_{\kk n}},
\eea
%
where $g_s\approx 2$ and we chose $e>0$. 

\section{{\tt berry\_task=-dos}: the density of states }

The density of states is calculated with the same width and type of smearing, as the other properties of the {\tt gyrotropic} module

\section{{\tt berry\_task=-noa}: the interband contributionto the natural optical activity }

Natural optical rotatory power is given by \cite{ivchenko-spss75}
%
\beq
\label{eq:rho-c}
\rho_0(\ww)=\frac{\ww^2}{2c^2}\re\,\gamma_{xyz}(\ww).
\eeq
%
for light propagating ling the main symmetry axis of a crystal $z$. Here $\gamma_{xyz}(\ww)$
is an anti-symmetric (in $xy$) tensor with units of length, which has both inter- and intraband contributions.

Following Ref.~\cite{malashevich-prb10} for the interband contribution we writewe write,
with $\partial_c\equiv\partial/\partial k_c$,
%
\begin{multline}
\re\,\gamma_{abc}^{\mathrm{inter}}(\ww)=\frac{e^2}{\varepsilon_0\hbar^2}
\int[d\kk]
\sum_{n,l}^{o,e}\,
\Bigl[ \frac{1}{\ww_{ln}^2-\ww^2} 
\re\left(A_{ln}^bB_{nl}^{ac}-A_{ln}^aB_{nl}^{bc}\right) \\
-\frac{3\ww_{ln}^2-\ww^2}{(\ww_{ln}^2-\ww^2)^2} 
\partial_c(E_l+E_n)\im\left(A_{nl}^aA_{ln}^b\right)   
\Bigr].
\label{eq:gamma-inter}
\end{multline}
%
The summations over $n$ and $l$ span the occupied ($o$) and empty
($e$) bands respectively, $\ww_{ln}=(E_l-E_n)/\hbar$,
and  ${\bm A}_{ln}(\kk)$ is given by (\ref{eq:berry-connection-matrix}) Finally, the matrix
$B_{nl}^{ac}$ has both orbital and spin contributions given by
%
\beq
\label{eq:B-ac-orb}
B_{nl}^{ac\,({\rm orb})}=
  \bra{u_n}(\partial_aH)\ket{\partial_c u_l}
 -\bra{\partial_c u_n}(\partial_aH)\ket{u_l}
\eeq
%
and
%
\beq
\label{eq:B-ac-spin}
B_{nl}^{ac\,({\rm spin})}=-\frac{i\hbar^2}{m_e}\epsilon_{abc}
\bra{u_n}\sigma_b\ket{u_l}.
\eeq
%
The spin matrix elements contribute less than 0.5\% of the total
$\rho_0^{\rm inter}$ of Te.  Expanding
$H=\sum_m \ket{u_m} E_m \bra{u_m}$ we obtain for the orbital matrix
elements
\beq
B_{nl}^{ac\,({\rm orb})}=-i\partial_a(E_n+E_l)A_{nl}^c \sum_m \Bigl\{ (E_n-E_m) A_{nm}^aA_{ml}^c -(E_l-E_m) A_{nm}^cA_{ml}^a \Bigr\}.
\label{eq:Bnl-sum}
\eeq
%
This reduces the calculation of $B^{\text{(orb)}}$ to the evaluation
of band gradients and off-diagonal elements of the Berry connection
matrix. Both operations can be carried out efficiently in a
Wannier-function basis following Ref.~\cite{yates-prb07}.


\section{{\tt berry\_task=-spin}: compute also the spin component of NOA and KME }

Unless this task is specified, only the orbital contributions are calcuated in NOA and KME, thus contributions from \eqs{m-spin}{B-ac-spin} are omitted.

