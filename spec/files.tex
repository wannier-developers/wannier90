\chapter{Files}


\section{seedname.win}
INPUT. The master input file; contains the specification of the system
and any parameters for the run. 

\subsection{Units}

The following are the dimensional quantities that are
specified in the master input file:

\begin{itemize}
\item Direct lattice vectors
\item Energy windows
\item Positions (of atomic or projection) centers in real space
\item Positions of $\mathbf{k}$-points in reciprocal space
\item \verb#zona# and \verb#box-size# (see Section~\ref{sec:proj})
\end{itemize}

Notes:

\begin{itemize}
\item The keyword \verb#length_unit# may be set to \verb#ang#
  (default) or \verb#bohr#, in order to set the units in which the
  direct  lattice vectors are given. 
\item Energy is always in eV.
\item Positions of $\mathbf{k}$-points are always in crystallographic
  coordinates
relative to the reciprocal lattice vectors.
\item \verb#box-size# and \verb#zona# always in Angstrom and
  reciprocal Angstrom, respectively
\end{itemize}

The reciprocal lattice vectors
$\{\mathbf{B}_{1},\mathbf{B}_{2},\mathbf{B}_{3}\}$ are defined in
terms
of the direct lattice vectors
$\{\mathbf{A}_{1},\mathbf{A}_{2},\mathbf{A}_{3}\}$ by the equation

\begin{equation}
\mathbf{B}_{1} = \frac{2\pi}{\Omega}\mathbf{A}_{2}\times\mathbf{A}_{3}
\ \ \ \mathrm{etc.},
\end{equation}

where the cell volume is
$\Omega=\mathbf{A}_{1}\cdot(\mathbf{A}_{2}\times\mathbf{A}_{3})$.

\section{seedname.mmn}
INPUT. See Chapter~\ref{ch:wann-pp}.

\section{seedname.amn}
INPUT. See Chapter~\ref{ch:wann-pp}.

\section{seedname.eig}
INPUT. See Chapter~\ref{ch:wann-pp}.

\section{seedname.nnkp} \label{sec:old-nnkp}
OUTPUT. See Chapter~\ref{ch:wann-pp}.

\section{seedname.wout}
OUTPUT. The master output file.


\section{seedname.dat}
INPUT/OUTPUT. Sufficient information to restart the calculation or enter the
plotting phase.

\section{WFNn.s}
INPUT. If \verb#overlap#=\verb#TRUE# or \verb#wann_plot#=\verb#TRUE# or \verb#slice_plot#=\verb#TRUE# 
. The orbitals from a first
principles calculation, indexed by kpoint n and spin s on a real space
grid. I can't imagine any code would output its wavefunctions
automatically in real space. Therefore one would always need to do
some coding and we would be able to specify the format of the files. A
header of the file should contain; the number of grid points in each
direction, the planewave cut-off energy, the total number of bands in
the file.






